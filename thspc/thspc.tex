\documentclass[12pt]{article}
\usepackage{xeCJK}
\usepackage{fontspec}
\usepackage[a4paper,top=2.5cm,bottom=2.5cm,left=2.2cm,right=2.2cm]{geometry}
\usepackage{graphicx}
\usepackage{listings}
\usepackage{xcolor}
\usepackage{indentfirst}
\usepackage{tikz}
\usepackage{amssymb}
\usepackage{amsthm}
\usepackage{amsmath}
\usepackage{tabularx}
\usepackage{hyperref}
\usepackage{ulem}
\usepackage{version}
\usepackage{thmtools}
\usepackage{multicol}
\usepackage{fancyhdr}
\usepackage{makecell}
\usepackage{svg}

\XeTeXlinebreaklocale "zh"
\XeTeXlinebreakskip = 0pt plus 1pt

\setCJKmainfont[AutoFakeBold,AutoFakeSlant]{kaiu.ttf}
\setmainfont[BoldFont={timesbd.ttf},SlantedFont={timesi.ttf}]{times.ttf}
\usetikzlibrary{arrows,decorations.markings,decorations.pathreplacing}

\lstset{
basicstyle=\ttfamily,
backgroundcolor=\color{white}
}

\begin{document}
\renewcommand{\headrulewidth}{0pt}
\cfoot{}
\lhead{}
\chead{}
\rhead{}

\pagestyle{fancy}
\setlength\parindent{24pt}
%\setlength\parskip{10pt}
% \setlength\columnseprule{0.5pt}
\newcommand{\parspace}{\vspace{10pt}}
\newcommand{\sectitle}[1]{\parspace\noindent\textbf{#1}}
% \raggedcolumns

\lhead{\footnotesize \textbf{2020 師大附中資訊科校隊培訓 第二次模擬競賽}}

\Large
\begin{center}
    \textbf{社團博覽會 (Exhibition)}
\end{center}
\normalsize

\sectitle{問題描述}

西元 2109 年,一年一度的社團博覽會又要開始了,每一個社團為了招收到優秀的學弟妹,無不拚盡全力準備,附中競程社也是如此。

今年社聯會提出了一個新的措施:社聯會決定了 $n$ 個宣傳點,編號為 $1$、$2$、……、$n$,分散在校園的各處,每個社團都可以在其中選擇三個,並且只能在這三個地方宣傳或擺設攤位,避免大家到處跑來跑去,造成混亂。

因為最近流感盛行,為了防疫,學務處只在 $n$ 個宣傳點之間開放了 $n-1$ 條道路,每條道路連接兩個相異的宣傳點,其他道路都被封鎖了,因此新生們只能利用這 $n-1$ 條路在各個宣傳點之間移動。保證對於任兩個相異的宣傳點 $u$、$v$,都可以沿著開放的道路從 $u$ 走到 $v$。

經過附中競程社的調查,得知如果以某個宣傳點為起點,往至少兩條不同的連接這個宣傳點的道路走,並且不再回到這個宣傳點,都可以到達同一個社團選擇的宣傳點,或是這個宣傳點已經被這個社團選擇了,在這裡的新生就會對這個社團感興趣。

舉例來說,以下兩張圖每個點都是宣傳點,線則表示道路,藍點是某個新生所在的位置,紅點是附中競程社所選擇的宣傳點,那麼左圖的新生不會對附中競程社感興趣,但右圖的會:

\begin{center}
    \includesvg[scale=0.5]{p2}
    \includesvg[scale=0.5]{p3}
\end{center}

附中競程社希望有盡量多的宣傳點,滿足新生在那個宣傳點時,會對附中競程社感興趣,例如以下圖中的三個紅點是附中競程社選擇的宣傳點,綠點則是除了這三個點外,在那裡的新生會對附中競程社感興趣的宣傳點(注意在附中競程社選擇的宣傳點上的新生,也會對附中競程社感興趣):

\begin{center}
    \includesvg[scale=0.5]{p4}
\end{center}

社長問身為附中競程社學術長的你,最多能有幾個宣傳點,滿足在那裡的新生會對附中競程社有興趣。

\clearpage

\sectitle{輸入格式}

第一行有一個整數 $n$($1 \leq n \leq 10^5$),表示社聯會決定了 $n$ 個宣傳點。

接下來有 $n-1$ 行,其中第 $i$ 行包含兩個整數 $u_i$、$v_i$($1 \leq u_i,v_i \leq n$),表示學務處開放了宣傳點 $u_i$ 和 $v_i$ 之間的道路。

\sectitle{輸出格式}

輸出一個整數,表示最多能有幾個宣傳點,滿足在那裡的新生會對附中競程社有興趣。

\begin{tabularx}{\linewidth}{|X|X|}
    \hline
    \makecell[lt]{
        \textbf{輸入範例 1}\\
        \lstinputlisting{s00_00.in}
    } &
    \makecell[lt]{
        \textbf{輸入範例 2}\\
        \lstinputlisting{s00_01.in}
    } \\ \hline
    \makecell[lt]{
        \textbf{輸出範例 1}\\
        \lstinputlisting{s00_00.out}
    } & 
    \makecell[lt]{
        \textbf{輸出範例 2}\\
        \lstinputlisting{s00_01.out}
    } \\ \hline
    \makecell[lt]{
        說明:\\
        最好的選擇是宣傳點 5、17、8。
    } & \\ \hline
\end{tabularx}

\sectitle{評分說明}

本題共有兩組測試資料,每組可有多筆測試資料:

第一組測試資料,開放的 $n-1$ 條道路構成一條路徑,共 8 分。

第二組測試資料,$n \leq 300$,共 23 分。

第三組測試資料,$n \leq 1000$,共 33 分。

第四組測試資料,無特殊限制,共 36 分。

\end{document}