\documentclass[12pt]{article}
\usepackage{xeCJK}
\usepackage{fontspec}
\usepackage[a4paper,top=2.8cm,bottom=2.8cm,left=2cm,right=2cm]{geometry}
\usepackage{graphicx}
\usepackage{listings}
\usepackage{xcolor}
\usepackage{indentfirst}
\usepackage{tikz}
\usepackage{amssymb}
\usepackage{amsthm}
\usepackage{amsmath}
\usepackage{fancyhdr}
\usepackage{tabularx}
\usepackage{hyperref}
\usepackage{ulem}
\usepackage{version}
\usepackage{thmtools}
\usepackage{qtree}
\usepackage{algorithm}
\usepackage{algpseudocode}
\usepackage{mathtools}
\usepackage{csquotes}
\usepackage{makecell}
\usepackage{multicol}

\XeTeXlinebreaklocale "zh"
\XeTeXlinebreakskip = 0pt plus 1pt

\setCJKmainfont[AutoFakeBold,AutoFakeSlant]{SourceHanSansTW-Regular.otf}
\usetikzlibrary{arrows,decorations.markings,decorations.pathreplacing}
\pagestyle{fancy}
\newenvironment{Hint}{\noindent\textbf{Hint.}}{}

\tikzstyle {graph node} = [circle, draw, minimum width=1cm]
\tikzset{edge/.style = {decoration={markings,mark=at position 1 with %
            {\arrow[scale=2,>=stealth]{>}}},postaction={decorate}}}

\lstset{
basicstyle=\ttfamily,
backgroundcolor=\color{white}
}

\declaretheoremstyle[
notefont=\bfseries,
notebraces={}{},
bodyfont=\normalfont\itshape,
headformat=\NAME~\NUMBER~\NOTE,
headpunct={\\},
spaceabove=12pt
]{problem}
\declaretheoremstyle[
notefont=,
notebraces={}{},
bodyfont=\normalfont,
headformat=\NAME~\NUMBER~\NOTE,
headpunct={},
spaceabove=12pt
]{Theorem}
\declaretheorem[style=problem, name=例題, numberwithin=section]{example}
\declaretheorem[style=problem, name=習題, numberwithin=section]{practice}
\declaretheorem[style=Theorem, name=Theorem, numberwithin=section]{Theorem}
\declaretheorem[style=Theorem, name=Lemma, numberwithin=section]{Lemma}
\declaretheorem[style=Theorem, name=Proof, numberwithin=section]{Proof}

\newcommand{\sectitle}[1]{\large\noindent\textbf{#1}\normalsize}
\newcommand{\subsectitle}[1]{\noindent\textbf{#1}}
\newcommand{\saminput}[1]{\noindent\textbf{範例輸入 #1}}
\newcommand{\samoutput}[1]{\noindent\textbf{範例輸出 #1}}
\newcommand{\samnote}[1]{\vspace{-12pt} \noindent 說明:#1}
\newcommand{\probtitle}[3]{
    \clearpage
    \begin{center}
        \huge
        #1. #2 \\
        \normalsize
        Problem ID: #3
    \end{center}
}
\raggedcolumns
\setlength\columnseprule{0.5pt}

\begin{document}

\renewcommand{\headrulewidth}{0pt}
\cfoot{}
\lhead{}
\chead{}
\rhead{}

\begin{titlepage}
    \vspace*{\fill}
    \begin{center}
        \huge
        師大附中電算社\\
        109 上學期社內賽

        \vspace{50pt}
        \Large
        請詳閱注意事項後再開始作答
        \normalsize
    \end{center}
    \vspace*{\fill}
\end{titlepage}

\clearpage

\renewcommand{\headrulewidth}{1pt}
\pagenumbering{arabic}
\setlength\parindent{24pt}
\setlength\parskip{12pt}
\cfoot{\thepage}
\lhead{師大附中電算社}
\rhead{109 上學期社內賽}

\noindent
\huge
\textbf{注意事項}
\normalsize

\begin{enumerate}
    \item 本測驗只能以 C++ 作答,作答語言請選擇以下其中之一:
    \begin{itemize}
        \item GNU G++11 7.3.0
        \item GNU G++14 7.3.0
        \item GNU G++17 7.3.0
        \item GNU G++17 9.2.0 (64 bit, msys 2)
    \end{itemize}
    \item 測驗期間可以上網查詢資料、查看紙本資料、舉手請學術組成員幫忙或與他人討論,唯不得抄襲他人程式碼。
    \item 每一題都有若干個子題,一個子題中有數筆測試資料,需通過全部測試資料才能得到該子題的分數。一題的分數以單一提交的最高分數計。
    \item 排名比序如下:
    \begin{enumerate}
        \item 總分較高者
        \item 滿分題數較多者
        \item 最後一次分數改變的時間較早者(依 submission 編號為準)
    \end{enumerate}
    \item 得到至少 100 分者可於 4 點後領取小獎品,高一前九名與高二前四名可領取大獎品。學術組成員不參與排名。
    \item 對題目敘述有疑問,請使用競賽系統的提問功能。
    \item 若覺得使用 \texttt{cin} 與 \texttt{cout} 輸入輸出消耗太多執行時間,請在 main 函數開頭加上:
\begin{lstlisting}
std::ios_base::sync_with_stdio(false);
std::cin.tie(0);
\end{lstlisting}
    並以 \texttt{'{\textbackslash}n'} 代替 \texttt{endl}。加上這些後請不要使用 \texttt{printf} 和 \texttt{scanf}。
    \item 題目有按照(學術長認為的)難度由易到難排序。
    \item $a \bmod b$ 指的是 $a$ 除以 $b$ 取餘數,也就是 \texttt{a \% b}。
    \item 在進行加、減、乘之前,先將數字 $\bmod$ 過並不改變結果。也就是:
    \begin{itemize}
        \item $(a + b) \bmod m = ((a \bmod m) + (b \bmod m)) \bmod m$
        \item $(a - b) \bmod m = ((a \bmod m) - (b \bmod m)) \bmod m$
        \item $(a \times b) \bmod m = ((a \bmod m) \times (b \bmod m)) \bmod m$
    \end{itemize}
\end{enumerate}

\clearpage

\probtitle{A}{巧克力鬆餅}{machine}
\input{chocolate.tex}

\probtitle{B}{簽到題}{checkin}
\input{checkin.tex}

\probtitle{C}{數學習作}{math}
\input{math.tex}

\probtitle{D}{北樓自助餐}{lunch}
\input{lunch.tex}

\probtitle{E}{評分表單}{scoring}
\input{scoring.tex}

\probtitle{F}{\Large 雞包包雞包包雞包紙包紙包雞包包雞紙包雞包紙包雞}{chicken}
\input{chicken.tex}

\probtitle{G}{神祕的資料結構}{mysterious}
\input{mysterious.tex}

\probtitle{H}{掃蕩}{clear}
\input{clear.tex}

\probtitle{I}{不插電的機器}{machine}
\input{machine.tex}

\probtitle{J}{我拆~我拆~我拆拆拆}{partition}
\input{partition.tex}

\end{document}