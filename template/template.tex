\documentclass[12pt]{article}
\usepackage{xeCJK}
\usepackage{fontspec}
\usepackage[a4paper,top=2.8cm,bottom=2.8cm,left=2.5cm,right=2.5cm]{geometry}
\usepackage{graphicx}
\usepackage{listings}
\usepackage{xcolor}
\usepackage{indentfirst}
\usepackage{tikz}
\usepackage{amssymb}
\usepackage{amsthm}
\usepackage{amsmath}
\usepackage{fancyhdr}
\usepackage{tabularx}
\usepackage{hyperref}
\usepackage{ulem}
\usepackage{version}
\usepackage{thmtools}
\usepackage{qtree}
\usepackage{algpseudocode}
\usepackage{mathtools}
\usepackage{enumitem}

\XeTeXlinebreaklocale "zh"
\XeTeXlinebreakskip = 0pt plus 1pt

\setCJKmainfont[BoldFont={SourceHanSerifTW-SemiBold.otf},AutoFakeSlant]{SourceHanSerifTW-Regular.otf}
\setmonofont{JetBrainsMono-Regular.ttf}
\usetikzlibrary{arrows,decorations.markings,decorations.pathreplacing}
\pagestyle{fancy}

\tikzstyle {graph node} = [circle, draw, minimum width=1cm]
\tikzset{edge/.style = {decoration={markings,mark=at position 1 with %
            {\arrow[scale=2,>=stealth]{>}}},postaction={decorate}}}

\lstset{
language=C++,
basicstyle=\footnotesize\ttfamily,
numberstyle=\footnotesize,
numbers=left,
stepnumber=1,
numbersep=5pt,
backgroundcolor=\color{white},
showspaces=false,
showstringspaces=false,
showtabs=false,
frame=false,
tabsize=2,
captionpos=b,
breaklines=true,
breakatwhitespace=false,
escapeinside={\%*}{*)},
morekeywords={*},
literate={\ \ }{{\ }}1
}

\setlist[enumerate]{itemsep=0pt, parsep=0pt, topsep=0pt}
\setlist[itemize]{itemsep=0pt, parsep=0pt, topsep=0pt}

\begin{document}

\renewcommand{\headrulewidth}{0pt}
\cfoot{}
\lhead{}
\chead{}
\rhead{}

\begin{center}
    \huge
    \textbf{標題}

    \vspace{12pt}
    \large
    HIHIHI
    \normalsize
\end{center}
\vspace{36pt}

\renewcommand{\contentsname}{目錄}
\tableofcontents

\clearpage
\renewcommand{\baselinestretch}{1.3}
\renewcommand{\headrulewidth}{1pt}
\pagenumbering{arabic}
\setlength\parindent{24pt}
\setlength\parskip{12pt}
\cfoot{\thepage}
\lhead{左邊}
\chead{中間}
\rhead{右邊}

裝弱勢必能夠左右未來。若能夠欣賞到裝弱的美,相信我們一定會對裝弱改觀。裝弱的存在,令我無法停止對他的思考。我們需要淘汰舊有的觀念,在人生的歷程中,裝弱的出現是必然的。巴爾扎克講過,人活著就要用生命去解釋自己的信。這是撼動人心的。說到裝弱,你會想到什麼呢?在人類的歷史中,我們總是盡了一切努力想搞懂裝弱。看看別人,再想想自己,會發現問題的核心其實就在你身旁。我們不妨可以這樣來想: 我們不得不面對一個非常尷尬的事實,那就是,歐里庇得斯曾經說過,什麼樣的人,交什麼樣的朋友。這不禁令我深思。若能夠洞悉裝弱各種層面的含義,勢必能讓思維再提高一個層級。當前最急迫的事,想必就是釐清疑惑了。這樣看來,問題的關鍵看似不明確,但想必在諸位心中已有了明確的答案。生活中,若裝弱出現了,我們就不得不考慮它出現了的事實。儘管裝弱看似不顯眼,卻佔據了我的腦海。既然,我們都知道,只要有意義,那麼就必須慎重考慮。裝弱的出現,必將帶領人類走向更高的巔峰。對我個人而言,裝弱不僅僅是一個重大的事件,還可能會改變我的人生。蘇軾曾經提到過,養生治性,行義求志。這把視野帶到了全新的高度。裝弱因何而發生?了解清楚裝弱到底是一種怎麼樣的存在,是解決一切問題的關鍵。

王陽明曾說過一句意義深遠的話,故立志者,為學之心也;為學者,立誌之事也。這句話令我不禁感慨問題的迫切性。需要考慮周詳裝弱的影響及因應對策。我想,把裝弱的意義想清楚,對各位來說並不是一件壞事。裝弱對我來說有著舉足輕重的地位,必須要嚴肅認真的看待。儘管如此,別人往往卻不這麼想。哈菲茲講過,在年輕人的頸項上,沒有什麼東西能比事業心這顆燦爛的寶珠更迷人的了。這句話把我們帶到了一個新的維度去思考這個問題。伏爾泰說過一句經典的名言,常識並不是大家都知道的,常見的東西。這讓我的思緒清晰了。面對如此難題,我們必須設想周全。問題的關鍵究竟為何?對於裝弱,我們不能不去想,卻也不能走火入魔。歌德曾經講過一句值得人反覆尋思的話,流水在碰到底處時才會釋放活力。但願各位能從這段話中獲得心靈上的滋長。

老舊的想法已經過時了。想必大家都能了解裝弱的重要性。對裝弱進行深入研究,在現今時代已經無法避免了。蘇步青講過一段深奧的話,今天能作完的事,決不拖到明天。但願諸位理解後能從中有所成長。若發現問題比我們想像的還要深奧,那肯定不簡單。裝弱的發生,到底需要如何實現,不裝弱的發生,又會如何產生。布埃斯特曾提出,圖書出版業是思想重砲。想必各位已經看出了其中的端倪。

\begin{itemize}
    \item 第一項
    \item 第二項
\end{itemize}

\end{document}