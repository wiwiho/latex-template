\documentclass[12pt]{article}
\usepackage{xeCJK}
\usepackage{fontspec}
\usepackage[a4paper,top=2.8cm,bottom=2.8cm,left=2.8cm,right=2.8cm]{geometry}
\usepackage{graphicx}
\usepackage{listings}
\usepackage{xcolor}
\usepackage{indentfirst}
\usepackage{tikz}
\usepackage{amssymb}
\usepackage{amsthm}
\usepackage{amsmath}
\usepackage{tabularx}
\usepackage{hyperref}
\usepackage{ulem}
\usepackage{version}
\usepackage{thmtools}
\usepackage{multicol}
\usepackage{fancyhdr}

\XeTeXlinebreaklocale "zh"
\XeTeXlinebreakskip = 0pt plus 1pt

\setCJKmainfont[AutoFakeBold,AutoFakeSlant]{kaiu.ttf}
\setmainfont[BoldFont={timesbd.ttf},SlantedFont={timesi.ttf}]{times.ttf}
\usetikzlibrary{arrows,decorations.markings,decorations.pathreplacing}

\lstset{
basicstyle=\ttfamily,
backgroundcolor=\color{white}
}

\begin{document}
\renewcommand{\headrulewidth}{0pt}
\cfoot{}
\lhead{}
\chead{}
\rhead{}

\pagestyle{fancy}
\setlength\parindent{24pt}
%\setlength\parskip{10pt}
\setlength\columnseprule{0.5pt}
\newcommand{\parspace}{\vspace{10pt}}
\newcommand{\sectitle}[1]{\parspace\noindent\textbf{#1}}
\raggedcolumns

\LARGE
\begin{center}
    202006 APCS 模擬測驗

    \vspace{15pt}
    2020.06.28
\end{center}
\normalsize

\vspace{80pt}

\Large
\begin{center}
    實作題 \hspace{10pt} 注意事項
\end{center}

\large
\vspace{10pt}

\noindent
注意事項與操作說明:\\
\href{https://hackmd.io/@joylintp/APCSM_202006}{https://hackmd.io/@joylintp/APCSM\_202006}
\begin{enumerate}
    \item 本測驗並非由官方舉辦,成績僅供參考,不應作為大學程式設計先修檢測之成績證明。
    \item 測驗過程中建議不要使用任何方式查詢與題目相關之資訊,或是使用預先準備好的電子資料,以達到最佳模擬測驗效果。
    \item 對於公開組,建議各語言選擇項目如下:
        \begin{itemize}
            \item C:GNU GCC C11 5.1.0
            \item C++:GNU G++11 5.1.0
            \item Java:Java 11.0.5
            \item Python:Python 3.7.2
        \end{itemize}
    \item 測驗中的評測結果正確僅代表通過範例測資。計算成績時只會以各題最後一筆的程式碼評測。
\end{enumerate}

\normalsize

\clearpage

\rhead{\footnotesize 109 年 6 月 28 日\\202006 APCS 模擬測驗}

\Large
\begin{center}
    \textbf{第 1 題 \hspace{12pt} 站名顯示器 (Monitor)}
\end{center}
\normalsize

\sectitle{問題描述}

小皓是認真上課的乖小孩,他每天都會準時搭捷運上學。今天在捷運他看著車廂中的站名顯示器,突然發現左右兩側的站名順序和箭頭方向竟然是相反的!

舉例來說,經過的車站名稱依序為 \texttt{ABC}、\texttt{DEF}、\texttt{GHI}、\texttt{JKL},當某側顯示器的內容為 \texttt{ABC->DEF->GHI->JKL},則另一側顯示的則會是 \texttt{JKL<-GHI<-DEF<-ABC}。如此一來,無論乘客面向哪一側的顯示器,上面顯示的箭頭指的方向便會與列車行進方向相同了。

小皓覺得這樣的設計太有趣了,他決定在家中走廊的兩側用相同的方式貼上所有捷運站的名字。他已經做好了其中一側的貼紙了,你能幫他印出另一側的文字嗎?

\sectitle{輸入格式}

輸入只有一行,為一個長度不超過 500 且不含空白的字串,為其中一側貼紙上的文字。其中車站名稱僅由大寫英文字母組成,並且名稱間會以 \texttt{->} 兩字元隔開。保證至少會有兩個車站,並且可能會有相同名稱的車站。

\sectitle{輸出格式}

將車站以相反的順序輸出,車站之間以 \texttt{<-} 兩字元隔開,字元間不應有多餘的空白。

\begin{multicols}{2}

\sectitle{範例一:輸入}

\begin{lstlisting}
ABC->DEF->GHI->JKL
\end{lstlisting}

\sectitle{範例一:輸出}

\begin{lstlisting}
JKL<-GHI<-DEF<-ABC
\end{lstlisting}

\sectitle{(說明)}

經過的車站名稱依序為 \texttt{ABC}、\texttt{DEF}、\texttt{GHI}、\texttt{JKL}。

\columnbreak

\sectitle{範例二:輸入}

\begin{lstlisting}
Y->S->A->E->S->I->S->C->P->A
\end{lstlisting}

\sectitle{範例二:輸出}

\begin{lstlisting}
A<-P<-C<-S<-I<-S<-E<-A<-S<-Y
\end{lstlisting}

\end{multicols}

\sectitle{評分說明:}

每一筆測試資料的執行時間限制均為 1 秒,記憶體限制均為 256 MB,依正確通過測資筆數給分。其中:

第 1 子題組 20 分:所有車站名稱皆只由一個大寫英文字母組成。

第 2 子題組 80 分:無額外限制。

\end{document}