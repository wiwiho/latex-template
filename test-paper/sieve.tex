\TestSection{題組 - 埃拉托斯特尼演算法}

對於一個大於 $1$ 的正整數 $p$,我們稱其為一個質數若其正因數只有 $1$ 和 $p$。若某個大於 $1$ 的正整數 $x$ 不是質數,則存在兩個正整數 $a,b\ (1 < a \le b < x)$ 使得 $x = a \times b$,且 $a$ 為不超過 $\sqrt{x}$ 的質數。因此,在判斷 $x$ 是否為質數時,只須使用介於 $2$ 到 $\lfloor\sqrt{x}\rfloor$ 之間的所有質數試除即可;若所有數都無法整除 $x$,則 $x$ 為一個質數。

埃拉托斯特尼演算法能在 $O(n\log\log n)$ 的時間複雜度內找出所有不超過 $n$ 的質數,其作法如下:先將所有介於 $2$ 到 $n$ 的整數標示為質數,並由小到大枚舉所有介於 $2$ 到 $\lfloor\sqrt{n}\rfloor$ 之間的正整數 $i$。若 $i$ 被標示為質數,則將所有 $i$ 的倍數($2$ 倍或以上)標示為非質數。枚舉結束後即可得知每個整數是否為質數。

以下示範使用埃拉托斯特尼演算法找出不超過 $25$ 的質數,其中刪除線表示該數已被標示為非質數:
\begin{itemize}
    \item 將介於 $2$ 到 $20$ 的整數標為質數:2 3 4 5 6 7 8 9 10 11 12 13 14 15 16 17 18 19 20 21 22 23 24 25
    \item 枚舉介於 $2$ 到 $\lfloor\sqrt{25}\rfloor = 5$ 的整數
    \begin{itemize}
        \item $2$ 被標示為質數,故將 $4,6,\ldots,20$ 標示為非質數:2 3 \sout{4} 5 \sout{6} 7 \sout{8} 9 \sout{10} 11 \sout{12} 13 \sout{14} 15 \sout{16} 17 \sout{18} 19 \sout{20} 21 \sout{22} 23 \sout{24} 25
        \item $3$ 被標示為質數,故將 $6,9,\ldots,24$ 標示為非質數:2 3 \sout{4} 5 \sout{6} 7 \sout{8} \sout{9} \sout{10} 11 \sout{12} 13 \sout{14} \sout{15} \sout{16} 17 \sout{18} 19 \sout{20} \sout{21} \sout{22} 23 \sout{24} 25
        \item $4$ 被標示為非質數
        \item $5$ 被標示為質數,故將 $5,10,\ldots,25$ 標示為非質數:2 3 \sout{4} 5 \sout{6} 7 \sout{8} \sout{9} \sout{10} 11 \sout{12} 13 \sout{14} \sout{15} \sout{16} 17 \sout{18} 19 \sout{20} \sout{21} \sout{22} 23 \sout{24} \sout{25}
    \end{itemize}
    \item 枚舉結束,不超過 $25$ 的質數有:2 3 5 7 11 13 17 19 23
\end{itemize}

右方為埃拉托斯特尼演算法的實作,其中 \texttt{isprime[i]} 若為 $1$ 表示 $i$ 被標示為質數;否則 $i$ 被標示為非質數,請回答第 16. 題及第 17. 題:

\begin{multicols}{2}

\begin{enumerate}[start=16]

\begin{TestProblem}

\TestBlank{1}、\TestBlank{2}、\TestBlank{3} 依序應填入:

\TestOptionOne
{\lstinline{i * i <= n};\lstinline{i * j <= n};\lstinline{i * j}}
{\lstinline{i <= (n ^ 1/2)};\lstinline{j <= n};\lstinline{j}}
{\lstinline{i * i <= n};\lstinline{j <= n};\lstinline{j}}
{\lstinline{i <= (n ^ 1/2)};\lstinline{i * j <= n};\lstinline{i * j}}

\end{TestProblem}

\begin{TestProblem}

右方程式碼輸出為何?

\TestOptionQuarter{\texttt{117}}{\texttt{74}}{\texttt{75}}{\texttt{113}}

\end{TestProblem}

\end{enumerate}

\columnbreak

\begin{lstlisting}
#include<stdio.h>

int main(){
    int n = 100;
    
    int isprime[101];
    for(int i = 2; i <= n; i++)
        isprime[i] = 1;

    int cnt = 0;
    for(int i = 2; __(1)__; i++)
        if(isprime[i])
            for(int j = 2; __(2)__; j++)
                isprime[__(3)__] = 0, cnt++;

    printf("%d\n", cnt);

    return 0;
}
\end{lstlisting}

\end{multicols}
