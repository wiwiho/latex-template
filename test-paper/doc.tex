\documentclass[12pt]{article}
\usepackage{xeCJK}
\usepackage{fontspec}
\usepackage[a4paper,top=1.5cm,bottom=2cm,left=1.5cm,right=1.5cm]{geometry}
\usepackage{graphicx}
\usepackage{listings}
\usepackage{xcolor}
\usepackage{indentfirst}
\usepackage{tikz}
\usepackage{amssymb}
\usepackage{amsthm}
\usepackage{amsmath}
\usepackage{fancyhdr}
\usepackage{tabularx}
\usepackage{hyperref}
\usepackage{ulem}
\usepackage{version}
\usepackage{thmtools}
\usepackage{qtree}
\usepackage{algpseudocode}
\usepackage{mathtools}
\usepackage{multicol}
\usepackage[shortlabels]{enumitem}
\usepackage{ctable}
\usepackage{totcount}
\usepackage{makecell}

\XeTeXlinebreaklocale "zh"
\XeTeXlinebreakskip = 0pt plus 1pt

\setCJKmainfont[AutoFakeBold,AutoFakeSlant]{Noto Serif CJK TC}
% \setmonofont{Consolas}
\newfontfamily{\TestTitleMainFont}{NotoSerifCJK-Bold.ttc}
\newCJKfontfamily{\TestTitleCJKFont}{NotoSansCJK-Bold.ttc}
\newcommand{\TestTitleFont}{\TestTitleMainFont\TestTitleCJKFont}

\pagestyle{fancy}

\lstset{
    language=C,
    basicstyle=\footnotesize\ttfamily,
    breaklines=true,
    % numberstyle=\footnotesize\ttfamily,
    % numbers=left,
    % numbersep=1pt,
    % stepnumber=1,
    breakatwhitespace=false,
    escapeinside={\%*}{*)},
    morekeywords={*},
    literate={\ \ }{{\ }}1,
    aboveskip=0.2cm,
    belowskip=0cm
}

\newcommand{\TestTitle}[1]{
    \begin{center}
        \TestTitleFont\LARGE{#1}
    \end{center}
}
\newcommand{\TestSection}[1]{\par\noindent{\TestTitleFont #1}\normalsize\par}

\newenvironment{TestProblem}{
    \item
}{}

\newcommand{\TestOptionQuater}[4]{
    \hspace{-0.28cm}
    \begin{tabular}{p{0.2\linewidth}p{0.2\linewidth}p{0.2\linewidth}p{0.2\linewidth}}
        (A) #1 & (B) #2 & (C) #3 & (D) #4
    \end{tabular}
}
\newcommand{\TestOptionHalf}[4]{
    \hspace{-0.28cm}
    \begin{tabular}{p{0.4\linewidth}p{0.4\linewidth}}
        (A) #1 & (B) #2 \\
        (C) #3 & (D) #4
    \end{tabular}
}
\newcommand{\TestOptionOne}[4]{
    \begin{enumerate}[(A)]
        \item #1
        \item #2
        \item #3
        \item #4
    \end{enumerate}
}

\newcommand{\TestBlank}[1]{\texttt{\_\_(#1)\_\_}}

\setlist[enumerate]{itemsep=0pt, parsep=0pt, topsep=0pt, leftmargin=*}
\setlist[itemize]{itemsep=0pt, parsep=0pt, topsep=0pt, leftmargin=*}

\raggedcolumns

\begin{document}

\renewcommand{\headrulewidth}{0pt}
\renewcommand{\baselinestretch}{1.1}
\pagenumbering{arabic}
\setlength\parindent{0pt}
\setlength\parskip{12pt}
\cfoot{\thepage}

\TestTitle{110 學年度~師大附中資訊學科能力競賽~筆試}

*題目中的程式碼皆以 C11 為標準

\TestSection{單選題(共 25 題,一題 4 分,總分 100 分)}

\begin{enumerate}
    \begin{TestProblem}

\begin{multicols}{2}

右方程式碼之輸出為何?

\TestOptionQuater{\texttt{65}}{\texttt{60}}{\texttt{54}}{\texttt{50}}

\columnbreak

\begin{lstlisting}
#include<stdio.h>

int main(){
    int T = 5, s = 0;
    while(T--)
        s += T + 10;
    printf("%d\n", s);

    return 0;
}
\end{lstlisting}

\end{multicols}

\end{TestProblem}
    \begin{TestProblem}

\begin{multicols}{2}

右方程式碼之輸出為何?

\hspace{-0.28cm}
\begin{tabular}{p{0.4\linewidth}p{0.4\linewidth}}

\begin{enumerate}[(A)]
\item ~
\begin{lstlisting}[aboveskip=-0.8\baselineskip, numbers=none]
-++++
--+++
---++
----+
\end{lstlisting}
\end{enumerate} &

\begin{enumerate}[(A), start=2]
\item ~
\begin{lstlisting}[aboveskip=-0.8\baselineskip, numbers=none]
+
-++
--+++
---++++
\end{lstlisting}
\end{enumerate} \\

\vspace{-0.8cm}
\begin{enumerate}[(A), start=3]
\item ~
\begin{lstlisting}[aboveskip=-0.8\baselineskip, numbers=none]
++++
-+++
--++
---+
\end{lstlisting}
\end{enumerate} &

\vspace{-0.8cm}
\begin{enumerate}[(A), start=4]
\item ~
\begin{lstlisting}[aboveskip=-0.8\baselineskip, numbers=none]
---++++
--+++
-++
+
\end{lstlisting}
\end{enumerate}

\end{tabular}

\columnbreak

\begin{lstlisting}
#include<stdio.h>

int main(){
    for(int i = 0; i <= 3; i++){
        for(int j = i; j > 0; j--)
            printf("-");
        for(int j = i; j < 4; j++)
            printf("+");
        printf("\n");
    }

    return 0;
}
\end{lstlisting}

\end{multicols}

\end{TestProblem}
    \begin{TestProblem}

\begin{multicols}{2}

右方程式碼之輸出為何?

\TestOptionQuarter{\texttt{36}}{\texttt{34}}{\texttt{39}}{\texttt{24}}

\columnbreak

\begin{lstlisting}
#include<stdio.h>

int a = 1;

int main(){
    int a = 2, ans = 0;
    for(int i = 0; i < a; i++){
        for(int j = 0; j < a; j++)
            ans += a;
        a = 3, ans += a;
    }
    ans += a;

    printf("%d\n", ans);

    return 0;
}
\end{lstlisting}

\end{multicols}

\end{TestProblem}
    \clearpage
    \begin{TestProblem}

\begin{multicols}{2}

右方程式碼之輸出為何?

\TestOptionQuarter{\texttt{12}}{\texttt{10}}{\texttt{11}}{\texttt{13}}

\columnbreak

\begin{lstlisting}
#include<stdio.h>

int f(int x){
    int ret = 0;
    switch(x){
        case 1:
            ret = 1;
            break;
        case 2:
            ret = 2;
        case 3:
            ret = 3;
            return ret;
        default:
            ret = 4;
            break;
    }
    return ret;
}

int main(){
    printf("%d\n", f(1) + f(2) + f(3) + f(4));

    return 0;
}
\end{lstlisting}

\end{multicols}

\end{TestProblem}
\end{enumerate}

\TestSection{題組 - 埃拉托斯特尼演算法}

對於一個大於 $1$ 的正整數 $p$,我們稱其為一個質數若其正因數只有 $1$ 和 $p$。若某個大於 $1$ 的正整數 $x$ 不是質數,則存在兩個正整數 $a,b\ (1 < a \le b < x)$ 使得 $x = a \times b$,且 $a$ 為不超過 $\sqrt{x}$ 的質數。因此,在判斷 $x$ 是否為質數時,只須使用介於 $2$ 到 $\lfloor\sqrt{x}\rfloor$ 之間的所有質數試除即可;若所有數都無法整除 $x$,則 $x$ 為一個質數。

埃拉托斯特尼演算法能在 $O(n\log\log n)$ 的時間複雜度內找出所有不超過 $n$ 的質數,其作法如下:先將所有介於 $2$ 到 $n$ 的整數標示為質數,並由小到大枚舉所有介於 $2$ 到 $\lfloor\sqrt{n}\rfloor$ 之間的正整數 $i$。若 $i$ 被標示為質數,則將所有 $i$ 的倍數($2$ 倍或以上)標示為非質數。枚舉結束後即可得知每個整數是否為質數。

以下示範使用埃拉托斯特尼演算法找出不超過 $25$ 的質數,其中刪除線表示該數已被標示為非質數:
\begin{itemize}
    \item 將介於 $2$ 到 $20$ 的整數標為質數:2 3 4 5 6 7 8 9 10 11 12 13 14 15 16 17 18 19 20 21 22 23 24 25
    \item 枚舉介於 $2$ 到 $\lfloor\sqrt{25}\rfloor = 5$ 的整數
    \begin{itemize}
        \item $2$ 被標示為質數,故將 $4,6,\ldots,20$ 標示為非質數:2 3 \sout{4} 5 \sout{6} 7 \sout{8} 9 \sout{10} 11 \sout{12} 13 \sout{14} 15 \sout{16} 17 \sout{18} 19 \sout{20} 21 \sout{22} 23 \sout{24} 25
        \item $3$ 被標示為質數,故將 $6,9,\ldots,24$ 標示為非質數:2 3 \sout{4} 5 \sout{6} 7 \sout{8} \sout{9} \sout{10} 11 \sout{12} 13 \sout{14} \sout{15} \sout{16} 17 \sout{18} 19 \sout{20} \sout{21} \sout{22} 23 \sout{24} 25
        \item $4$ 被標示為非質數
        \item $5$ 被標示為質數,故將 $5,10,\ldots,25$ 標示為非質數:2 3 \sout{4} 5 \sout{6} 7 \sout{8} \sout{9} \sout{10} 11 \sout{12} 13 \sout{14} \sout{15} \sout{16} 17 \sout{18} 19 \sout{20} \sout{21} \sout{22} 23 \sout{24} \sout{25}
    \end{itemize}
    \item 枚舉結束,不超過 $25$ 的質數有:2 3 5 7 11 13 17 19 23
\end{itemize}

右方為埃拉托斯特尼演算法的實作,其中 \texttt{isprime[i]} 若為 $1$ 表示 $i$ 被標示為質數;否則 $i$ 被標示為非質數,請回答第 16. 題及第 17. 題:

\begin{multicols}{2}

\begin{enumerate}[start=16]

\begin{TestProblem}

\TestBlank{1}、\TestBlank{2}、\TestBlank{3} 依序應填入:

\TestOptionOne
{\lstinline{i * i <= n};\lstinline{i * j <= n};\lstinline{i * j}}
{\lstinline{i <= (n ^ 1/2)};\lstinline{j <= n};\lstinline{j}}
{\lstinline{i * i <= n};\lstinline{j <= n};\lstinline{j}}
{\lstinline{i <= (n ^ 1/2)};\lstinline{i * j <= n};\lstinline{i * j}}

\end{TestProblem}

\begin{TestProblem}

右方程式碼輸出為何?

\TestOptionQuarter{\texttt{117}}{\texttt{74}}{\texttt{75}}{\texttt{113}}

\end{TestProblem}

\end{enumerate}

\columnbreak

\begin{lstlisting}
#include<stdio.h>

int main(){
    int n = 100;
    
    int isprime[101];
    for(int i = 2; i <= n; i++)
        isprime[i] = 1;

    int cnt = 0;
    for(int i = 2; __(1)__; i++)
        if(isprime[i])
            for(int j = 2; __(2)__; j++)
                isprime[__(3)__] = 0, cnt++;

    printf("%d\n", cnt);

    return 0;
}
\end{lstlisting}

\end{multicols}


\end{document}