\documentclass[12pt]{article}
\usepackage{xeCJK}
\usepackage{fontspec}
\usepackage[a4paper,top=2.8cm,bottom=2.8cm,left=2.3cm,right=2.3cm]{geometry}
\usepackage{graphicx}
\usepackage{listings}
\usepackage{xcolor}
\usepackage{indentfirst}
\usepackage{tikz}
\usepackage{amssymb}
\usepackage{amsthm}
\usepackage{amsmath}
\usepackage{fancyhdr}
\usepackage{tabularx}
\usepackage{hyperref}
\usepackage{ulem}
\usepackage{version}
\usepackage{thmtools}
\usepackage{qtree}
\usepackage{algpseudocode}
\usepackage{mathtools}
\usepackage{multicol}
\usepackage{enumitem}
\usepackage{ctable}
\usepackage{totcount}

\XeTeXlinebreaklocale "zh"
\XeTeXlinebreakskip = 0pt plus 1pt

\setCJKmainfont[BoldFont={SourceHanSerifTW-SemiBold.otf},AutoFakeSlant]{SourceHanSerifTW-Regular.otf}
\setmonofont{Consolas}
\newfontfamily{\ProblemTitleMainFont}{SourceHanSansTW-Bold.otf}
\newCJKfontfamily{\ProblemTitleCJKFont}{SourceHanSansTW-Bold.otf}
\newcommand{\ProblemTitleFont}{\ProblemTitleMainFont\ProblemTitleCJKFont}

\pagestyle{fancy}

\lstset{
basicstyle=\footnotesize\ttfamily
}

% \raggedcolumns

\newcommand{\ProblemTitle}[3]{\noindent\Huge{\ProblemTitleFont #1. #2 (#3)}\normalsize\par}
\newcommand{\ProblemSection}[1]{\vspace{0.6cm}\par\noindent\large{\ProblemTitleFont #1}\normalsize\par}
\newcommand{\ProblemSubsection}[1]{\par\noindent{\ProblemTitleFont #1}\par}
\newcommand{\ProblemStatement}{\ProblemSection{問題敘述}}
\newcommand{\ProblemInput}{\ProblemSection{輸入說明}}
\newcommand{\ProblemOutput}{\ProblemSection{輸出說明}}
\newcommand{\ProblemConstraints}{\ProblemSection{測資限制}}

\newcommand{\ProblemSampleTitle}{\ProblemSection{範例測資}}

\newcounter{ProblemSample}
\newcommand{\ProblemSample}[2]{
    \stepcounter{ProblemSample}
    \begin{multicols}{2}
        \ProblemSubsection{範例輸入 \theProblemSample}
        \lstinputlisting{#1}
        \columnbreak
        \ProblemSubsection{範例輸出 \theProblemSample}
        \lstinputlisting{#2}
    \end{multicols}
}
\newenvironment{ProblemSampleWithNote}[2]{
    \stepcounter{ProblemSample}
    \begin{multicols}{2}
        \ProblemSubsection{範例輸入 \theProblemSample}
        \lstinputlisting{#1}
        \columnbreak
        \ProblemSubsection{範例輸出 \theProblemSample}
        \lstinputlisting{#2}
    \end{multicols}
    \vspace{-0.4cm}
    \ProblemSubsection{範例說明 \theProblemSample}
}{}

\newcommand{\ProblemSubtaskTitle}{\ProblemSection{評分說明}}
\newtotcounter{ProblemSubtask}
\newenvironment{ProblemSubtaskTable}{
    \begin{center}
        \begin{tabular}{ccl}
            \specialrule{1.3pt}{0pt}{1pt}
            子任務 & 分數 & 額外輸入限制 \\
            \specialrule{0.5pt}{1pt}{1pt}
}
{
            \specialrule{1.3pt}{1pt}{0pt}
        \end{tabular}
    \end{center}
}
\newcommand{\ProblemSubtask}[2]{ \stepcounter{ProblemSubtask} \theProblemSubtask & #1 & #2 \\ }

\setlist[enumerate]{itemsep=0pt, parsep=0pt, topsep=0pt}
\setlist[itemize]{itemsep=0pt, parsep=0pt, topsep=0pt}

\begin{document}

\renewcommand{\headrulewidth}{0pt}
\renewcommand{\baselinestretch}{1.3}
\pagenumbering{arabic}
\setlength\parindent{24pt}
\setlength\parskip{12pt}
\cfoot{\thepage}
\rhead{\small{2021 師大附中暑期資訊培訓 \\ 模擬競賽 I}}

\ProblemTitle{D}{刷怪塔}{Monster}

\ProblemStatement

在知名遊戲 Minceraft 中,玩家可以每擊殺一隻怪物即可獲得 $1$ 點經驗值,並且在經驗值累積到一定點數後便可以升級。詳細的說,一名累積有 $x$ 點經驗值的玩家等級為 $l$,其中 $l$ 為滿足 $x \ge \frac{l(l+1)}{2}$ 的最大非負整數。

在遊戲中,玩家可以建造一種特殊的建築「刷怪塔」來生成大量怪物並將其困在指定的安全範圍內,此時玩家即可在不受傷害的情況下擊殺怪物並獲得大量經驗值以提升等級。當玩家達到一定等級便可透過「附魔」來強化武器的傷害或增強盔甲的防禦力等。

櫻巫女剛剛又再次因為跌入岩漿而損失了自己身上所有的裝備及經驗值。為了附魔重新合成的武器,她來到了兩座刷怪塔前準備擊殺怪物並獲得經驗值。一開始第一座刷怪塔中有 $a$ 隻怪物,而第二座刷怪塔中有 $b$ 隻。

為了避免獲得多餘的經驗值,她打算使用以下的方式擊殺怪物:選擇目前較多怪物的刷怪塔(數量相同則選擇第一座),設櫻巫女仍須獲得 $x$ 點經驗值才能提升至下一個等級,若該刷怪塔仍有至少 $x$ 隻怪物,櫻巫女便會擊殺其中的恰好 $x$ 隻,並重新選擇刷怪塔;若該刷怪塔中怪物不足 $x$ 隻則停止擊殺怪物。

刷怪塔的建造者不知火芙蕾雅擔心櫻巫女再次跌入岩漿中而浪費了資源,她想先請你幫忙算算,在使用了上述的方式擊殺怪物後,櫻巫女的等級會是多少,以及兩座刷怪塔中分別還剩下多少怪物。

\ProblemInput

輸入包含 $T$ 筆測資。每筆測資僅一行,包含兩個整數 $a,b$,表示兩座刷怪塔一開始的怪物數量。

\ProblemOutput

對於每筆測資,輸出一行包含三個非負整數 $l,a',b'$,表示在擊殺怪物後櫻巫女的等級、第一座刷怪塔中剩下的怪物、第二座刷怪塔中剩下的怪物。

\clearpage

\ProblemConstraints

\begin{itemize}
    \item $1 \le T \le 10^4$
    \item $0 \le a,b \le 10^{18}$
\end{itemize}

\ProblemSampleTitle

\begin{ProblemSampleWithNote}{01.in}{01.out}
    以第三筆測資進行說明,櫻巫女擊殺怪物的順序如下:
    \begin{itemize}
        \item 擊殺第二座刷怪塔中的 $1$ 隻怪物,兩座刷怪塔分別剩下 $7$ 和 $11$ 隻怪物
        \item 擊殺第二座刷怪塔中的 $2$ 隻怪物,兩座刷怪塔分別剩下 $7$ 和 $9$ 隻怪物
        \item 擊殺第二座刷怪塔中的 $3$ 隻怪物,兩座刷怪塔分別剩下 $7$ 和 $6$ 隻怪物
        \item 擊殺第一座刷怪塔中的 $4$ 隻怪物,兩座刷怪塔分別剩下 $3$ 和 $6$ 隻怪物
        \item 擊殺第二座刷怪塔中的 $5$ 隻怪物,兩座刷怪塔分別剩下 $3$ 和 $1$ 隻怪物
    \end{itemize}
\end{ProblemSampleWithNote}

\ProblemSample{01.in}{01.out}

\ProblemSubtaskTitle

本題共有 \total{ProblemSubtask} 組子任務,條件限制如下所示。

\begin{ProblemSubtaskTable}
    \ProblemSubtask{8}{$a=0$}
    \ProblemSubtask{21}{$a=b$}
    \ProblemSubtask{12}{$a,b \le 1000$}
    \ProblemSubtask{59}{無額外限制}
\end{ProblemSubtaskTable}

\end{document}